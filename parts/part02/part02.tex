
\part{The Data Model}
\label{part:data-model}

\begin{partintro}
Git's core insight. Everything builds on this. Four object types, content-addressable storage, the DAG. This is the foundation.
\end{partintro}

\chapter{Content-Addressable Storage}
% Hash as identity (SHA-1, SHA-256 transition)
% Why this solves problems
% Automatic deduplication
% Integrity verification
% Implementation in object database

\chapter{The Four Object Types}
% Blob: raw file content
% Tree: directory structure  
% Commit: snapshot + metadata + parents
% Tag: annotated references
% Object format and serialization
% How they connect
% Why these four

\chapter{The Commit Graph (DAG)}
% Why a graph not a tree
% Parent relationships
% Merge commits (multiple parents)
% Graph properties that matter
% Implications for algorithms

\chapter{References and HEAD}
% Branches as movable pointers
% Tags (lightweight and annotated)
% HEAD and symbolic refs
% Remote tracking branches
% Packed refs optimization
% Reference transactions and atomicity

\chapter{The Index (Staging Area)}
% What it actually is
% Binary format
% Stat cache for performance
% Multiple stages (merge conflicts)
% Extensions: cache-tree, split-index
% Why three-way merge needs this

\chapter{Repository Layout}
% .git/ directory structure
% objects/ and refs/
% hooks/, info/, logs/
% config, HEAD, index
% What each component does