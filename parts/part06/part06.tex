
\part{Performance Engineering}
\label{part:performance}

\begin{partintro}
Git is fast by design. Specific optimizations, careful profiling, constant attention to performance.
\end{partintro}

\chapter{Operation Complexity}
% Branch: O(1) - why
% Commit: O(changed files)
% Merge: depends on merge base distance
% Log: O(commits traversed)
% Blame: O(n log n)
% Status: optimizations for large repos

\chapter{Caching Strategies}
% Object cache
% Parsed commit cache
% Tree cache in index
% Stat cache
% Name hash cache
% Delta base cache
% Invalidation strategies

\chapter{Parallelization}
% Pack compression threads
% Checkout parallelization
% Index preload
% Parallel directory traversal
% Thread pool management
% Where parallelism helps

\chapter{I/O Optimization}
% Memory-mapped files
% Pack window management
% Prefetching strategies
% Batched operations
% Reducing system calls

\chapter{Network Protocol}
% Pack protocol versions (v0, v1, v2)
% Negotiation phase (want/have)
% Finding common commits
% Thin packs
% Minimizing transfer
% Resumable operations

\chapter{Geometric Repacking}
% Why exponential pack sizes
% Reducing write amplification
% Maintenance strategy
% gc.bigPackThreshold
% Long-term efficiency